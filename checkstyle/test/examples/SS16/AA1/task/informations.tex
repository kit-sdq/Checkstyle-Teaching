\section*{Allgemeine Hinweise}

\subsection*{Erfolgskontrollen}
\begin{itemize}
	\item Das Bestehen der Programmieren-Lehrveranstaltung setzt zwei Erfolgskontrollen voraus:
	\item \textbf{Übungsblätter}
	\begin{itemize}
		\item Die erste Erfolgskontrolle erfordert das Erwerben des \textbf{Übungsscheins}.
		Dies ist durch das Erhalten von \textbf{mindestens 50\%} der maximal zu
		erreichenden Punktzahl aus den Übungsblättern möglich.
		\item Des Weiteren ist zum Erhalt eines Übungsscheins eine Anmeldung im
		\textbf{Campus Management Portal}\footnote{\url{https://campus.studium.kit.edu/}} erforderlich. Die Anmeldefrist endet am \textbf{10.02.2016}. Eine zeitnahe Anmeldung ist jedoch empfohlen.
		\item Der Übungsschein ist unbenotet.
	\end{itemize}
	\item \textbf{Abschlussaufgaben}
	\begin{itemize}	
		\item Die zweite Erfolgskontrolle stellt das Bestehen \textbf{zweier
		Abschlussaufgaben} dar und ist lediglich nach dem Erwerb der ersten
		Erfolgskontrolle möglich. Nähere Informationen zu den Abschlussaufgaben werden
		noch bekannt gegeben.
		\item Nach dem Erhalt eines Übungsscheines ist eine weitere Anmeldung zur
		Teilnahme an den Abschlussaufgaben im Campus Management Portal erforderlich. Die Anmeldefrist endet hierfür am \textbf{24.02.2016}.
		\item Die Gesamtnote der Programmieren-Lehrveranstaltung setzt sich alleinig aus den Noten der zwei Abschlussaufgaben zusammen.
	\end{itemize}
	\item Die Prüfung dieser Lehrveranstaltung gilt als
	\textbf{Orientierungsprüfung}. Dies setzt u.a. das Bestehen der
	Lehrveranstaltung bis Ende des dritten Semesters voraus. Zudem muss der erste
	Versuch in den ersten zwei Semestern geschehen.
\end{itemize}

\subsection*{Übungsblätter}
\begin{itemize}
		\item Im Rahmen dieser Veranstaltung werden insgesamt \textbf{6 Übungsblätter}, mit je 20 Punkten im Mittel, ausgegeben.
	\item Die Übungsblätter werden im Zweiwochenrhythmus auf der
	\textbf{Vorlesungshomepage}\footnote{\url{https://sdqweb.ipd.kit.edu/wiki/Programmieren}} veröfentlicht.
	\item Sie werden bis zum jeweiligen Abgabetermin das notwendige Wissen zum Lösen des Aufgabenblattes aus dem behandelten Lernstoff der Vorlesung und/oder dem Tutorium erhalten.
\end{itemize}
	
\subsection*{Das Praktomat-System}
\begin{itemize}
	\item Die Lösungen aller Aufgaben sind ausschließlich online über das
	\textbf{Praktomat-System}\footnote{\url{https://praktomat.cs.kit.edu/2015_WS/}}
	abzugeben. Abgaben auf anderem Weg werden grundsätzlich abgelehnt. Machen Sie sich daher rechtzeitig mit dem Praktomat-System vertraut und planen Sie eine Einarbeitungsphase.
	\item Ihre abgegebenen Lösungen werden vom Praktomat-System automatisch geprüft. Der Praktomat erzwingt eine Mindestqualität der Abgabe, indem Abgaben abgelehnt werden, die bestimmte \textbf{Qualitätskriterien} verletzen. 
	\item In den \textbf{Bearbeitungshinweisen} sind für jedes Übungsblatt alle Qualitätskriterien, welche Ihre Lösung einhalten sollte, aufgelistet. Diese Punkte werden im Laufe der Veranstaltung kontinuierlich ergänzt und erweitert, sodass Sie bei der Abgabe eines Blattes sowohl die bisherigen Tests, als auch die Tests im jeweiligen Blatt bestehen müssen.
	\item Wenn eine Abgabe vom Praktomaten abgelehnt wurde, kann Ihr Tutor diese auch nicht korrigieren. Planen Sie daher bei jeder Abgabe genügend Zeit ein, mindestens aber einen Tag Puffer. 
	\item Sie erhalten eine Zusammenfassung der Korrektur per E-Mail an Ihre KIT--Mailadresse. Alle Anmerkungen des Tutors können Sie anschließend im Praktomaten online einsehen.
	\item Der Zugriff auf das Praktomat-System ist ausschließlich im internen KIT-Netz möglich. Benutzen Sie, wenn Sie Zuhause arbeiten, den \textbf{VPN-Client} des KIT\footnote{\url{https://www.scc.kit.edu/dienste/vpn.php}}. 
	\item Bei Problemen bezüglich VPN wenden Sie sich an das SCC.
	\item Bei Problemen bezüglich dem Praktomat-System schreiben Sie Ihre Fragen in
	das ILIAS--Forum\footnote{\url{https://ilias.studium.kit.edu/goto_produktiv_crs_453424.html}}.
	\item Zur Abgabe Ihrer Lösungen wird jeweils eine Woche nach dem Erscheinen eines neuen Übungsblatt ein Praktomat-Task freigeschaltet.
% 	\item Um Ihre Lösungen im Praktomat-System abgeben zu können, müssen Sie eine
% 	\textbf{Einverständniserklärung} zur elektronischen Speicherung und Nutzung
% 	Ihrer Daten bis spätestens Sonntag, den 25. November abgeben. Ihre
% 	personalisierte Einverständniserklärung müssen Sie
% 	online\footnote{\url{https://sdqweb.ipd.kit.edu/disclaimer/}} erzeugen,
% 	anschließend ausdrucken und unterschrieben in den Programmieren-Briefkasten im
% 	1. Untergeschoss des Gebäudes 50.34 (Info-Bau) einwerfen.
\end{itemize}

\subsection*{Plagiarismus}
\begin{itemize}
	\item Die Übungsblätter sind selbständig zu bearbeiten. Das Einreichen fremder
	Lösungen, seien es auch teilweise Lösungen von Dritten, aus Büchern, dem
	Internet oder anderen Quellen gilt als \textbf{Täuschungsversuch}. Auch die
	Beihilfe, wie die Weitergabe der eigenen Lösung oder deren Teile, wird
	als Täuschungsversuch gewertet.
	\item Beim ersten Täuschungsversuch wird das komplette Übungsblatt mit 0 Punkten bewertet. Beim zweiten Täuschungsversuch wird der Übungsschein mit \textbf{\glqq nicht bestanden (5,0)\grqq} bewertet.
\end{itemize}

\subsection*{Tutorien}
\begin{itemize}
	\item Zusätzlich zur Vorlesung finden wöchentlich Tutorien statt, die Ihnen sowohl eine Nachbereitung von vergangenen Vorlesungen, als auch eine Besprechung von Übungsblättern anbieten. 
	\item Die Tutorien dienen als Ergänzung zum Vorlesungsstoff und gleichzeitig zur Vorbereitung der Abschlussaufgaben.
	\item Die Korrektur und Bewertung Ihrer abgegebenen Lösungen zu den
	Übungsblättern werden von Ihrem Tutor durchgeführt.
\end{itemize}

\subsection*{Kommunikation und aktuelle Informationen}
\begin{itemize}
	\item Fragen zu Vorlesungsinhalten und Übungsblättern stellen Sie bitte
	ausschließlich in den
	\textbf{ILIAS--Foren}\footnote{\url{https://ilias.studium.kit.edu/goto_produktiv_crs_453424.html}}.
	Achten Sie diesbezüglich darauf, dass Sie Ihre Fragen unter dem passenden Thema
	im Forum stellen. So profitieren auch Ihre Kommilitonen davon. E-Mails mit inhaltlichen Fragen werden aus genau diesem Grund nicht beantwortet, auch nicht von Tutoren.
	\item Auf der Vorlesungshomepage\footnote{\url{https://sdqweb.ipd.kit.edu/wiki/Programmieren}} veröffentlichen wir gelegentlich wichtige Neuigkeiten. Diese sind auch per Twitter abonnierbar. Eventuelle Korrekturen von Aufgabenstellungen werden auf diesem Weg bekannt gemacht. Das Beobachten der Neuigkeiten wird daher vorausgesetzt.
	\item Überprüfen Sie zudem das Postfach Ihrer \textbf{KIT--Mailadresse}
	regelmäßig auf neue E-Mails.
\end{itemize}
