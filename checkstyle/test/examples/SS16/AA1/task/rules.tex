\addtocounter{footnote}{1}\footnotetext{Der Praktomat wird die Abgabe zurückweisen, falls diese Regel verletzt ist.}\newcounter{required}\setcounter{required}{\value{footnote}}
\begin{itemize}
\item Achten Sie darauf nicht zu lange Zeilen, Methoden und Dateien zu erstellen\footnotemark[\value{required}]
\item Programmcode muss in englischer Sprache verfasst sein
\item Kommentieren Sie Ihren Code angemessen: So viel wie nötig, so wenig wie möglich
\item Wählen Sie geeignete Sichtbarkeiten für Ihre Klassen, Methoden und Attribute
\item Verwenden Sie keine Klassen der Java"=Bibliotheken ausgenommen Klassen der Pakete \texttt{java.lang}, \texttt{java.io} und \texttt{java.util}, es sei denn die Aufgabenstellung erlaubt ausdrücklich weitere Pakete\footnotemark[\value{required}]
\item Achten Sie auf fehlerfrei kompilierenden Programmcode\footnotemark[\value{required}]
\item Halten Sie alle Whitespace-Regeln ein\footnotemark[\value{required}]
\item Halten Sie die Regeln zu Variablen-, Methoden und Paketbenennung ein und wählen Sie aussagekräftige Namen\footnotemark[\value{required}]
\item Halten Sie die Regeln zu Javadoc-Dokumentation ein\footnotemark[\value{required}]
\item Nutzen Sie nicht das default-Package\footnotemark[\value{required}]
\item Halten Sie auch alle anderen Checkstyle-Regeln ein%\footnotemark[\value{required}]
\item System.exit und Runtime.exit dürfen nicht verwendet werden\footnotemark[\value{required}]
\end{itemize}

