%\vfill
\begin{tcolorbox}
\subsection*{Terminal-Klasse}
Laden Sie für \textbf{diese Aufgabe} die \texttt{Terminal}-Klasse\footnote{https://sdqweb.ipd.kit.edu/lehre/SS16-Programmieren/Terminal.java} von unserer Homepage herunter und platzieren Sie diese unbedingt im Paket \texttt{edu.kit.informatik}.
Die Methode \texttt{Terminal.readLine()} liest eine Benutzereingabe von der Konsole und ersetzt \texttt{System.in}.
Die Methode \texttt{Terminal.printLine()} schreibt eine Ausgabe auf die Konsole und ersetzt \texttt{System.out}.
Verwenden Sie für jegliche Konsoleneingabe oder Konsolenausgabe die \texttt{Terminal}-Klasse.
Verwenden Sie in keinem Fall \texttt{System.in} oder \texttt{System.out}.

Fehlermeldungen werden ausschließlich über die Terminal-Klasse ausgegeben und müssen aus technischen Gründen unbedingt mit \code{Error,} beginnen.

%Sie können davon ausgehen, dass alle Eingabe, bis auf explizit genannte Fälle bei den Befehlen, immer gültig sind. 
%In diesem Blatt müssen Sie noch keine Exceptions behandeln.

Laden Sie die Terminal-Klasse niemals zusammen mit Ihrer Abgabe hoch.
\end{tcolorbox}

%\vfill

