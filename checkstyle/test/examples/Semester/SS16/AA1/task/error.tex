%\vfill
\begin{tcolorbox}
\subsection*{Fehlerbehandlung}

Ihre Programme sollen auf ungültige Benutzereingaben mit einer aussagekräftigen Fehlermeldung reagieren. Aus technischen Gründen muss eine Fehlermeldung unbedingt mit \code{Error,} beginnen. Sie können hierzu die entsprechende Methode der Terminal-Klasse aufrufen. Eine Fehlermeldung führt nicht dazu, dass das Programm beendet wird; es sei denn, die nachfolgende Aufgabenstellung verlangt dies ausdrücklich. Achten Sie insbesondere auch darauf, dass unbehandelte \texttt{RuntimeExceptions}, bzw. Subklassen davon—sogenannte \emph{Unchecked Exceptions}—nicht zum Abbruch des Programms führen.

\end{tcolorbox}

%\vfill

