\documentclass{sdqassignment}

\lecture{Programmieren}
\semester{Sommersemester 2019}
\lecturer{Prof.\,Dr.\,Ralf H. Reussner}
\group{Software Design and Quality (SDQ)}
\ilias{https://sdqweb.ipd.kit.edu/wiki/Programmieren}
\mail{programmieren-vorlesung@ipd.kit.edu}
\assignment{Übungsblatt 3}
\points{20 Punkte}
\releasedate{22.05.2019, ca.\,13:00 Uhr}
\praktomatdate{29.05.2019, 13:00 Uhr}
\duedate{06.06.2019, 06:00 Uhr}
\version{Version 1.0}

\begin{document}


% Ein wichtiges Ziel dieser Aufgabe ist es, dass Sie eine saubere objektorientierte Modellierung planen und umsetzen. Überlegen Sie sich, welche Klassen hierzu nötig sind und welche Verbindungen zwischen ihnen bestehen müssen. Achten Sie dabei auf die Trennung von Ein-/Ausgabe und Funktionalität sowie die anderen aus der Vorlesung bekannten Kriterien für gutes Design. Halten Sie sich an die Java Code Conventions und dokumentieren Sie Ihre Lösung sinnvoll mit Hilfe von javadoc Kommentaren.


\hinweise{3}[\lstinlinejava{java.lang}, \lstinlinejava{java.io} und \lstinlinejava{java.util}]

\section*{Abgabemodalitäten}
Die Praktomat-Abgabe wird am \textbf{Mittwoch, den 29.05.2019 um 13:00 Uhr}, freigeschaltet. Achten Sie unbedingt darauf, Ihre Dateien im Praktomaten bei der richtigen Aufgabe vor Ablauf der Abgabefrist hochzuladen.

\begin{itemize}
    \item Geben Sie Ihre Klassen zu Aufgabe A als \texttt{*.java}-Dateien ab.
\end{itemize}

\terminalinfo

\newpage

\task{Hausaufgaben-Verwaltungssystems}{20}
In dieser Aufgabe sollen Sie die Grundfunktionalität eines Systems zur Verwaltung von Hausaufgaben implementieren. In einem solchen System gibt es verschiedene Arten von Benutzern: Schüler und Lehrer. Es werden Hausaufgaben verwaltet, welche von Schülern gelöst und von Lehrern korrigiert werden. Jede abgegebene Lösung kann höchstens von einem Lehrer korrigiert werden.

Zu Beginn enthält Ihr System weder vordefinierte Benutzer noch Hausaufgaben. Diese werden über eine textuelle Kommandoschnittstelle erzeugt. Dabei soll es möglich sein
\begin{itemize}
    \item Schüler und Lehrer anzulegen, wobei jeder Schüler immer genau einem Lehrer zugeordnet ist,
    \item Hausaufgaben anzulegen,
    \item eine Lösung zu einer bestimmten Hausaufgabe für einen bestimmten Schüler einzureichen,
    \item eine Korrektur für eine abgegebene Lösung zu erstellen,
    \item eine Notenübersicht aller korrigierten Lösungen je Hausaufgaben zu sehen,
    \item eine Zusammenfassung der Korrekturergebnisse je Hausaufgaben zu sehen,
    \item eine Zusammenfassung der Korrekturergebnisse je Lehrer zu sehen.
\end{itemize}

Beachten Sie dabei, dass ein Schüler zu jeder Hausaufgabe nur einmal eine Lösung einreichen darf. Ein Nachbessern der Lösung darf also nicht möglich sein. Die Korrektur einer Lösung soll hingegen beliebig oft überschrieben werden dürfen. Das System merkt sich nur die zuletzt abgegebene Korrektur zu einer Lösung.

\subsection{Interaktive Benutzerschnittstelle}
Da wir automatische Tests Ihrer interaktiven Benutzerschnittstelle durchführen, müssen die Ausgaben exakt den Vorgaben entsprechen. Insbesondere sollen sowohl Klein- und Großbuchstaben als auch die Leerzeichen und Zeilenumbrüche genau übereinstimmen. Geben Sie auch keine zusätzlichen Informationen aus. Beginnen Sie frühzeitig mit dem Einreichen, um Ihre Lösung dahingehend zu testen, und verwenden Sie das Forum, um eventuelle Unklarheiten zu klären.

\subsubsection{Formate der Ein- und Ausgaben}
Bei der Ein- und Ausgabe werden verschiedene Daten übergeben, beispielsweise Namen oder Personenkennzeichen. Entspricht eine Eingabe nicht dem hier vorgegebenen Format, dann ist immer eine Fehlermeldung auszugeben. Danach soll das Programm auf die nächste Eingabe warten.
\begin{itemize}
    \item Die Namen von Benutzern bestehen nur aus Kleinbuchstaben ohne Leerzeichen. Lehrer müssen immer unterschiedliche Namen haben, Schüler nicht. Mehrere Schüler dürfen denselben Namen haben. Ein Schüler darf auch denselben Namen wie ein Lehrer haben.
    \item Schüler werden über ihr Personenkennzeichen identifiziert, dieses ist eine positive 6-stellige ganze Zahl ohne führende Nullen.
    \item Der Text von Hausaufgaben, Lösungen und Korrekturen ist eine Zeichenkette, die keine Leerzeichen, Tabulatoren und Zeilenumbrüche enthalten darf. Verwenden Sie \lstinlinetxt{_} (Unterstrich) anstelle von Leerzeichen. Diese Einschränkung soll Ihnen das Einlesen der Befehlszeile erleichtern.
    \item Eine Note ist eine Zahl mit ganzzahligem Wert zwischen und inklusiv 1 und 6.
\end{itemize}

\subsubsection{Befehle}
Nach dem Start nimmt Ihr Programm über die Konsole mittels \lstinlinejava{Terminal.readLine()} Eingaben, welche im Folgenden näher spezifiziert werden. Nach Abarbeitung einer Eingabe wartet Ihr Programm auf weitere Eingaben, bis das Programm irgendwann durch die Eingabe der Zeichenfolge \lstinlinetxt{quit} beendet wird.

Achten Sie darauf, dass durch Ausführung der folgenden Befehle die zuvor definierten Grundlagen und Bedingungen nicht verletzt werden und geben Sie in diesen Fällen immer eine aussagekräftige Fehlermeldung aus. Eine Fehlermeldung führt nicht dazu, dass das Programm beendet wird.
% Achten Sie auch darauf, dass unbehandelte Ausnahmen (\lstinlinejava{Exceptions}) nicht zum Abbruch Ihres Programms führen.

Beachten Sie, dass bei der Beschreibung der Eingabe- und Ausgabeformate die Wörter zwischen spitzen Klammen (\lstinlinetxt{<} und \lstinlinetxt{>}) für Platzhalter stehen, welche bei der konkreten Ein- und Ausgabe durch Werte ersetzt werden. Diese eigentlichen Werte enthalten bei der Ein- und Ausgabe keine spitzen Klammern. Vergleichen Sie hierzu auch den Beispielablauf.

\subsubsection{Der Lehrer"=Befehl}
Legt einen neuen Lehrer mit Namen \lstinlinetxt{<name>} an und wählt diesen aus. Falls ein Lehrer mit diesem Namen bereits existiert, dann wird dieser ausgewählt und kein neuer Lehrer angelegt.
\paragraph{Eingabeformat}\lstinlinetxt{teacher <name>}

\subsubsection{Der Schüler"=Befehl}
Legt einen neuen Schüler mit Namen \lstinlinetxt{<name>} und Personenkennzeichen \lstinlinetxt{<pupilId>} an. Der Schüler wird dem zuletzt ausgewählten Lehrer zugewiesen.
\paragraph{Eingabeformat}\lstinlinetxt{pupil <name> <pupilId>}
\paragraph{Ausgabeformat}
Gibt eine Fehlermeldung aus und der Schüler wird nicht angelegt, falls es bereits einen Schüler mit der gleichen Personenkennzeichen gibt oder falls noch kein Lehrer ausgewählt wurde.

\subsubsection{Der Aufgaben"=Befehl}
Legt eine neue Aufgabe mit dem Text \lstinlinetxt{<text>} an. Der Aufgabe wird automatisch ein Identifikator zugewiesen. Dieser Identifikator ist eine positive natürliche Zahl, mit deren Hilfe die Aufgabe eindeutig identifiziert werden kann. Diese startet bei 1 und wird je angelegter Aufgabe um 1 erhöht.
\paragraph{Eingabeformat}\lstinlinetxt{assignment <text>}
\paragraph{Ausgabeformat}
Der Identifikator \lstinlinetxt{<assignmentId>} der erzeugten Aufgabe wird dabei ausgegeben: \lstinlinetxt{assignment id(<assignmentId>)}

\subsubsection{Der Schüler"=Auflisten"=Befehl}
Gibt eine Liste aller Schüler, aufsteigend numerisch sortiert nach Personenkennzeichen, aus.
\paragraph{Eingabeformat}\lstinlinetxt{list-pupils}
\paragraph{Ausgabeformat}
Dabei ist \lstinlinetxt{<name>} der Name des Schülers, \lstinlinetxt{<pupilId>} dessen Personenkennzeichen und \lstinlinetxt{<teacherName>} der Name des Lehrers, der den Schüler betreut. In jeder Zeile wird ein Schüler in folgendem Format ausgegeben:

\lstinlinetxt{(<pupilId>,<name>): <teacherName>}

\subsubsection{Der Abgeben"=Befehl}
Reicht eine Lösung mit dem Text \lstinlinetxt{<text>} zu der Aufgabe mit dem Identifikator \lstinlinetxt{<assignmentId>} für den Schüler mit der Personenkennzeichen \lstinlinetxt{<pupilId>} ein.
\paragraph{Eingabeformat}\lstinlinetxt{submit <assignmentId> <pupilId> <text>}
\paragraph{Ausgabeformat}
Gibt eine Fehlermeldung aus, wenn es die Aufgabe oder den Schüler nicht gibt oder wenn für den Schüler bereits eine Lösung zu dieser Aufgabe eingereicht wurde.

\subsubsection{Der Korrektur"=Befehl}
Erstellt eine Korrektur für die Lösung der Aufgabe mit dem Identifikator \lstinlinetxt{<assignmentId>} des Schüler mit der Personenkennzeichen \lstinlinetxt{<pupilId>}. Die Korrektur enthält die Note \lstinlinetxt{<grade>} und einen Text \lstinlinetxt{<text>}. Der Ersteller der Korrektur ist automatisch der Lehrer, der den betreffenden Schüler betreut.
\paragraph{Eingabeformat}\lstinlinetxt{review <assignmentId> <pupilId> <grade> <text>}
\paragraph{Ausgabeformat}
Dabei ist \lstinlinetxt{<teacherName>} der Name des Lehrers, der den Schüler betreut, \lstinlinetxt{<name>} der Name des Schüler, \lstinlinetxt{<pupilId>} dessen Personenkennzeichen und \lstinlinetxt{<grade>} die Note der Korrektur. Falls der Schüler, die Aufgabe oder die Lösung nicht existieren wird eine Fehlermeldung ausgegeben. Das Überschreiben einer existierenden Korrektur ist aber erlaubt. Wenn die Korrektur erfolgreich erstellt oder überschrieben wurde, dann ist folgende Meldung auszugeben:

\lstinlinetxt{<teacherName> reviewed (<pupilId>,<name>) with grade <grade>}

\subsubsection{Der Lösungen"=Auflisten"=Befehl}
Listet alle eingereichten Lösungen für die Aufgabe mit dem Identifikator \lstinlinetxt{<assignmentId>} auf. Die Ausgabe ist dabei numerisch aufsteigend nach dem Personenkennzeichen des Schülers sortiert.
\paragraph{Eingabeformat}\lstinlinetxt{list-solutions <assignmentId>}
\paragraph{Ausgabeformat}
Dabei ist \lstinlinetxt{<name>} der Name des Schülers, \lstinlinetxt{<pupilId>} dessen Personenkennzeichen und \lstinlinetxt{<text>} der Text der eingereichten Lösung.

\lstinlinetxt{(<pupilId>,<name>): <text>}

\subsubsection{Der Ergebnisse"=Befehl}
Gibt die Resultate für alle Aufgaben, numerisch aufsteigend sortiert nach ihrem Identifikator, aus.
\paragraph{Eingabeformat}\lstinlinetxt{results}
\paragraph{Ausgabeformat}
Dabei ist \lstinlinetxt{<pupilId>} die Personenkennzeichen des betreffenden Schüler und \lstinlinetxt{<grade>} die Note, mit der seine Lösung bewertet wurde. Je Aufgabe ist dabei folgendes auszugeben:
\begin{itemize}
    \item Eine Kopfzeile mit dem Identifikator \lstinlinetxt{<assignmentId>} und dem Text der Aufgabe \lstinlinetxt{<assignmentText>} im Format:

    \lstinlinetxt{assignment id(<assignmentId>): <assignmentText>}

    \item Eine Liste der bewerteten Lösungen, numerisch aufsteigend sortiert nach Personenkennzeichen. Eingereichte Lösungen ohne Bewertung sollen ignoriert werden. Jede Zeile der Ausgabe ist wie folgt zu formatieren:

    \lstinlinetxt{<pupilId>: <grade>}
\end{itemize}

\subsubsection{Der Aufgaben"=Auflisten"=Befehl}
Gibt eine Zusammenfassung der Resultate der Aufgaben, aufsteigend sortiert nach ihrem Identifikator, aus.
\paragraph{Eingabeformat}\lstinlinetxt{summary-assignment}
\paragraph{Ausgabeformat}
Dabei ist \lstinlinetxt{<assignmentId>} der Identifikator einer Aufgabe, \lstinlinetxt{<submitted>} ist die Anzahl der eingereichten Lösungen und \lstinlinetxt{<reviewed>} ist die Anzahl der korrigierten Lösungen. Je Aufgabe ist dabei folgendes auszugeben:

\lstinlinetxt{assignment id(<assignmentId>): <reviewed> / <submitted>}

\subsubsection{Der Lehrer"=Auflisten"=Befehl}
Gibt Informationen zu den einzelnen Lehrern aus. Die Ausgabe wird alphabetisch aufsteigend nach dem Namen des Lehrers sortiert.
\paragraph{Eingabeformat}\lstinlinetxt{summary-teacher}
\paragraph{Ausgabeformat}
Dabei ist \lstinlinetxt{<name>} der Name des Lehrers, \lstinlinetxt{<pupils>} die Anzahl der betreuten Schüler und \lstinlinetxt{<todo>} die Anzahl der noch zu bewertenden Lösungen, d.h. die Anzahl der eingereichten, aber noch nicht korrigierten Lösungen (aller Aufgaben) der Schüler, die der Lehrer betreut. Je Lehrer ist folgendes auszugeben:

\lstinlinetxt{<name>: <pupils> pupils, <todo> missing review(s)}

\subsubsection{Der Zurücksetzen"=Befehl}
Initialisiert das Verwaltungssystem neu. Der Zustand des Verwaltungssystems ist danach wie bei einem Neustart des Java"=Programms.
\paragraph{Eingabeformat}\lstinlinetxt{reset}

\subsubsection{Der Beenden"=Befehl}
Beendet das Verwaltungssystem. Hierfür dürfen keine Methoden wie \lstinlinejava{System.exit()} oder \lstinlinejava{Runtime.exit()} verwendet werden.
\paragraph{Eingabeformat}\lstinlinetxt{quit}


\subsection{Beispiel eines Programmablaufs}
Beachten Sie auch, dass bei dem folgenden Beispielablauf die Eingabezeilen mit dem \lstinlinetxt{>}"=Zeichen gefolgt von einem Leerzeichen eingeleitet werden. Diese beiden Zeichen sind ausdrücklich kein Bestandteil des eingegebenen Befehls, sondern dienen nur der Unterscheidung zwischen Ein- und Ausgabezeilen.

\lstinputsequence{Beispielablauf}

\end{document}
