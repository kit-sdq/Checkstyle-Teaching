\addtocounter{footnote}{1}\footnotetext{Der Praktomat wird die Abgabe zurückweisen, falls diese Regel verletzt ist.}\newcounter{required}\setcounter{required}{\value{footnote}}
\section*{Bewertungshinweise}
\begin{itemize}
	\item Achten Sie darauf, nicht zu lange Zeilen, Methoden und Dateien zu erstellen.\footnotemark[\value{required}]
	\item Programmcode muss in englischer Sprache verfasst sein.
	\item Kommentieren Sie Ihren Code angemessen: So viel wie nötig, so wenig wie möglich. Die Kommentare sollen einheitlich in englischer oder in deutscher Sprache verfasst werden.
	\item Wählen Sie geeignete Sichtbarkeiten für Ihre Klassen, Methoden und Attribute.
	\item Verwenden Sie keine Klassen der Java-Bibliotheken ausgenommen Klassen der Pakete \texttt{java.lang}, \texttt{java.util} und \texttt{java.util.regex}.\footnotemark[\value{required}]
	\item Achten Sie auf fehlerfrei kompilierenden Programmcode.\footnotemark[\value{required}]
	\item Halten Sie alle Whitespace-Regeln ein.\footnotemark[\value{required}]
	\item Halten Sie die Regeln zu Variablen-, Methoden- und Paketbenennung ein und wählen Sie aussagekräftige Namen.\footnotemark[\value{required}]
	\item Halten Sie die Regeln zur Javadoc-Dokumentation ein.\footnotemark[\value{required}]
	\item Nutzen Sie nicht das \texttt{default}-Package.\footnotemark[\value{required}]
	\item Halten Sie auch alle anderen Checkstyle-Regeln ein. Prinzipiell kann ein Nichteinhalten der Checkstyle-Regeln zu Punktabzug führen.
	\item \texttt{System.exit} und \texttt{Runtime.exit} dürfen nicht verwendet werden.\footnotemark[\value{required}]
\end{itemize}

\section*{Abgabemodalitäten}

Die Praktomat"=Abgabe wird am \textbf{Mittwoch, den 23. Januar 2019 um 13:00 Uhr}, freigeschaltet.

\begin{itemize}
	\item Geben Sie Ihre Klassen zu Aufgabe~A als \texttt{*.java}"=Dateien ab.
\end{itemize}
\textbf{Achten Sie unbedingt darauf, Ihre Dateien im Praktomaten vor Ablauf der Abgabefrist hochzuladen. Planen Sie für die Abgabe ausreichend Zeit ein, sollte der Praktomat Ihre Abgabe wegen einer Regelverletzung ablehnen. Wichtig: Laden Sie die Terminal"=Klasse nicht mit hoch.}


\begin{tcolorbox}
	\textbf{Checkstyle}
	\newline Denken Sie daran, den für dieses Blatt gültigen Checkstyle"=Regelsatz, welcher im ILIAS bereitgestellt wird, herunterzuladen.
	Bei dem Nichteinhalten des required"=Regelsatzes wird der Praktomat Ihre Abgabe zurückweisen.
	Das Nichteinhalten des optional"=Regelsatzes kann zu Punktabzug führen.
	Planen Sie, wie immer, ausreichend Zeit für die Abgabe ein, für den Fall, dass der Praktomat Ihre Abgabe wegen einer Regelverletzung ablehnt.
\end{tcolorbox}

\begin{tcolorbox}
	\textbf{Terminal"=Klasse}
	\newline Laden Sie für dieses Übungsblatt die \texttt{Terminal}"=Klasse, welche im ILIAS bereitgestellt wird, herunter und Platzieren Sie diese unbedingt im Paket \texttt{edu.kit.informatik}.
	Die Methode \texttt{Terminal.readLine()} liest eine Benutzereingabe von der Konsole und ersetzt somit \texttt{System.in.readLine()}.
	Die Methode \texttt{Terminal.printLine()} schreibt eine Ausgabe auf die Konsole und ersetzt \texttt{System.out.printLine()}.
	Verwenden Sie für jegliche Konsoleneingabe oder -ausgabe die \texttt{Terminal}"=Klasse.
	Verwenden Sie in keinem Fall Methoden aus \texttt{System.in} oder \texttt{System.out}.
	\textbf{Modifizieren Sie niemals die \texttt{Terminal}"=Klasse und laden Sie diese auch niemals zusammen mit Ihrer Abgabe hoch.}
\end{tcolorbox}

\begin{tcolorbox}
	\textbf{Öffentliche Tests}
	\newline Bitte beachten Sie, dass das erfolgreiche Bestehen der öffentlichen Tests für eine erfolgreiche Abgabe dieses Blattes notwendig ist. Planen Sie entsprechend Zeit für Ihren ersten Abgabeversuch ein.
\end{tcolorbox}

\begin{tcolorbox}
	\textbf{Objektorientierte Modellierung}
	\newline Achten Sie darauf, dass Ihre Abgaben sowohl in Bezug auf objektorientierte Modellierung als auch Funktionalität bewertet werden. Achten Sie daher auch auf eine Trennung von Benutzerinteraktion und Programmlogik.
\end{tcolorbox}
